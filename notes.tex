%*********************************************
\addcontentsline{toc}{section}{¿Cómo publicar en RECANEWS?}
\section*{¿Cómo publicar en RECANEWS?}
%*********************************************
Las dos posibles alternativas para publicar en\par RECANEWS son:\\
\\
\textbf{I.} Se puede realizar un pull request a en el repositorio github en volumen del mes:
\begin{center}
 \url{https://github.com/recanews/2015_4}
\end{center}

\noindent \textbf{II.} Enviar un correo a \url{reca.news@gmail.com} con las siguientes especificaciones:
\begin{description}
\item[Correo:]\url{reca.news@gmail.com}
\item[Asunto:]Título de la publicación
\item[Contenido:]Contenido de la publicación\\
Persona encargada (Opcional)\\
Correo de la persona encargada (Opcional)
\end{description}
  

%*********************************************
\addcontentsline{toc}{section}{¿Quién recibe RECANEWS?}
\section*{¿Cómo publicar en RECANEWS?}
%*********************************************

Cualquier persona interesada en hacer parte de la base de datos puede enviar un correo a 
\begin{description}
\item[Correo:]\url{reca.news@gmail.com}
\item[Asunto:]Subcripción a RECANEWS
\item[Contenido:]Correo (Obligatorio).\\
Nombre o nickname (opcional)
\end{description}

%*********************************************
\addcontentsline{toc}{section}{Repositorio de RECANEWS}
\section*{Repositorio de RECANEWS}
%*********************************************

Repositorio: \url{https://github.com/recanews}\\



%*********************************************
\addcontentsline{toc}{section}{Contacto del RECA}
\section*{Contacto del RECA:}
%*********************************************

\begin{description}
\item[Correo RECA:]\url{reca.astronomia@gmail.com}
\item[Facebook:] \url{https://www.facebook.com/RECAstronomia}
\item[Google$+$:] \url{http://goo.gl/P0DEf4}
\item[Twitter:] \url{https://twitter.com/RECAstronomia}
\item[Pagina Web:] \url{http://goo.gl/Fl4zQP}
\end{description}


%*********************************************
\addcontentsline{toc}{section}{Representantes de RECA en las regiones}
\section*{Representantes de RECA en las regiones}
%*********************************************
\begin{description}
\item[Medellín:]Malory Agudelo Vásquez\\
\url{magudelov@gmail.com}\\ \url{magudelov@fisica.udea.edu.co}
\item[Pereira:]Luisa Fernanda Cardona\\ \url{luisferncardona@utp.edu.co}
\item[Pasto:]Katherine Mafla Oliva\\
\url{skmofis13@gmail.com}
\item[Cali:]Daniel Santacruz\\
\url{santacruz121@gmail.com}
\item[Bogotá:]Maria Camila Remolina Gutierrez\\
\url{mc.remolina197@uniandes.edu.co}\\

Andres Felipe Ramos Padilla\\
\url{andresrp25@gmail.com}\\

Juan David Jimenez Nieto\\
\url{jdjimenezn@correo.udistrital.edu.co}
\end{description}


%*********************************************
\addcontentsline{toc}{section}{Editor RECANEWS}
			\section*{Editor RECANEWS}
%********************************************
  
\begin{flushright}
J. Sebastián Castellanos-Durán\\
\url{jscastellanosd@unal.edu.co}\\
Observatorio Astronómico Nacional\\
Universidad Nacional de Colombia
\end{flushright}
\begin{flushright}
Cualquier comentario por favor escribir al correo  \url{reca.news@gmail.com}\\
Mayo 2015
\end{flushright}


